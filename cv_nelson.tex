%-----------------------------------------------------------------------
% Anderson Konzen (anderson.konzen at gmail)
% Template of a CV typeset in XeTeX
% 
% URL: https://github.com/anderkonzen/xetexcv
% DISCLAIMER: This template is provided for free and without any
%             guarantee that it will correctly compile on your
%             system if you have a non-standard configuration.
%
% This work is a fork from https://github.com/dartar/cvtex. This
% modified version is licensed under the same Creative Commons
% Attribution-ShareAlike license:
% http://creativecommons.org/licenses/by-sa/3.0/.
%
% (Please, keep a link to https://github.com/anderkonzen/xetexcv on
% your source file for attribution as per CC-license.)
%-----------------------------------------------------------------------

%!TEX TS-program = xelatex
%!TEX encoding = UTF-8 Unicode

\documentclass[10pt, letter]{article}

\def\myname{Nelson Chen}
\def\myemail{nelson@mindflakes.com}
% \def\myskype{crazysim1}
\def\mycellphone{(626) 723-3427}
\def\myaddress{Santa Barbara, CA}


%% Loads the fontspec package, so we can easily select system fonts
%\usepackage{fontspec}
%% It implements some odds-and-ends features and improved functionality for broken or sub-standard LaTeX methods when using the XeTeX format. Already loads the fontspec package.
\usepackage{xltxtra}
%% Loads xcolor package, so we can use color names instead of rgb values
\usepackage{xcolor}

%-----------------------------------------------------------------------
% Document Layout
%-----------------------------------------------------------------------
%% Loads geometry package, so we can easily change page setup
\usepackage{geometry}
\geometry{letterpaper,         %% Specifies the paper size
  textwidth=7.5in,         %% Specifies the width of body (the text area)
  textheight=10.0in,        %% Specifies the height of body (the text area)
  marginparsep=7pt,        %% Modifies separation between body and marginal notes
  marginparwidth=.5in}     %% Modifies width of the marginal notes
\setlength\parindent{0in}  %% Amount of indentation at the first line of a paragraph.

\usepackage{multicol}

%-----------------------------------------------------------------------
% Fonts
%-----------------------------------------------------------------------
\setmainfont[Mapping={tex-text},Numbers={OldStyle},Ligatures={Common}]{Linux Libertine O}
\setsansfont[Mapping={tex-text},Color=AA0000]{Linux Biolinum O}
\setmonofont[Mapping={tex-text},Scale=0.8]{Menlo}

%-----------------------------------------------------------------------
% Custom Commands
%-----------------------------------------------------------------------
\chardef\&="E050  % Redefine '&' character
\newcommand{\html}[1]{\href{#1}{\scriptsize\textsc{[html]}}}
\newcommand{\pdf}[1]{\href{#1}{\scriptsize\textsc{[pdf]}}}
\newcommand{\doi}[1]{\href{#1}{\scriptsize\textsc{[doi]}}}
%% Configure margin years
\usepackage{marginnote}
\newcommand{\years}[1]{\marginnote{#1}}
\renewcommand*{\raggedleftmarginnote}{}  %% Define margin note alignment (in this case, justified text at the left margin)
%\setlength{\marginparsep}{7pt}  %% Already defined in geometry
\reversemarginpar  %% Margin notes in left side of the page

\renewcommand{\years}[1]{{\emph{#1}}}

% Reference
\newcommand{\reference}[4]{
\subsection*{#1 | {\footnotesize{#2}}}
#3 -- \href{mailto:#4}{\texttt{#4}}
}

\newenvironment{packed_enum}{
\begin{enumerate}
  \setlength{\itemsep}{1pt}
  \setlength{\parskip}{0pt}
  \setlength{\parsep}{0pt}
}{\end{enumerate}}

\newenvironment{packed_item}{
\begin{itemize}
  \setlength{\itemsep}{1pt}
  \setlength{\parskip}{0pt}
  \setlength{\parsep}{0pt}
}{\end{itemize}}

%-----------------------------------------------------------------------
% Headings
%-----------------------------------------------------------------------
%% Provides a set of commands for changing the font used for the various sectional headings
\usepackage{sectsty}
%% Provides various types of underlining that can stretch between words and be broken across lines
\usepackage[
  normalem  %% \em and \emph still produce normal italics
]{ulem}
%% Change font size of section headers
\sectionfont{\color[HTML]{AA0000}\mdseries\upshape\Large}
\subsectionfont{\mdseries\scshape\normalsize} 
\subsubsectionfont{\mdseries\upshape\large} 

%-----------------------------------------------------------------------
% PDF Setup
%-----------------------------------------------------------------------
\usepackage[
  xetex,            %% Use XeTeX backend
  colorlinks=true,  %% Colors the text of links and anchors
  breaklinks=true,  %% Allow links to break over lines
  pdftitle={{\myname} - Curriculum Vitae},
  pdfauthor={\myname}
]{hyperref}  
\hypersetup{linkcolor=blue,filecolor=black,urlcolor=blue}


%-----------------------------------------------------------------------
%-----------------------------------------------------------------------
% DOCUMENT
%-----------------------------------------------------------------------
\begin{document}
{\LARGE \myname}\\
\hrule
\begin{multicols}{2}
\vspace{0.2in}
{\large phone: \mycellphone}\\[.05cm]
{\large email: \href{mailto:\myemail}{\texttt{\myemail}}}\\[.05cm]
% {\large skype: \myskype}\\[.2cm]
\myaddress
%\vfill
% \vspace{-0.2in}

\section*{Education}

\years{2008-2012}

\textsc{B.Sci} in Computer Science at University of California, Santa Barbara - Santa Barbara, CA

Expected Graduation: Fall 2012

\subsection*{Relevant Courses taken at UCSB:}

\begin{packed_item}
    \item CS130 -- C++ Data Structures and Analysis
    \item CS154 -- Computer Architecture
    \item CS160 -- Computer Languages
\end{packed_item}

%\hrule
\section*{Experience}

\subsection*{UCSB Housing Residential Network | {\footnotesize{Santa Barbara, CA}}}

Network Consultant

\years{2009- \ldots} 

Helped residents in UCSB Residential Housing get their devices online with ResNet.
Developed tools to streamline configuration of these devices for printing to
shared printers and to automatically configure the network. Helped physically
install switches and hardware in dozens of locations on campus. Internal
documentation with regards to networking and procedures was also maintained by
me. Assisted users one-on-one in troubleshooting any connectivity issues both
simple and complex. Produced automated setup tools for all major platforms to
assist in configuration of user systems.

\subsection*{UCSB Associated Students | {\footnotesize{Santa Barbara, CA}}}

Technical Committee Vice Chair

\years{2009-2012}

Managed technical aspects of many committee activities including its website
and software.

Designed, setup, configure, and maintained a VMWare, Linux, Windows, and
pfSense-based network, accompanying virtual machines for file sharing and
network traffic management, and devised and improved upon workarounds for
problematic network issues for LAN Parties for 150+ attendees for every
quarter since Fall of 2009.

% \subsection*{Because of Hope (Nonprofit)| {\footnotesize{Santa Barbara, CA}}}

% Technical Support

% \years{2012- \ldots}

% Maintain existing website and VPS. Drafting and advising plans to help
% organization save money on website hosting and increase website performance

\section*{Selected Personal Projects and Contributions}

\subsection*{UC Santa Barbara Subreddit Design | {\footnotesize{Santa Barbara, CA}}}

Creator and Maintainer

\years{May 2012 - \ldots}

Produced most advanced Python and Ruby framework and deployment tool for
developing preprocessed stylesheets and templated markdown for section of
website subscribed to by 1,700 users. Said framework is easy to deploy with
intuitive and minimal configuration and utilizes accepted standard Python and
Ruby support tools. Framework is easy to extend should more advanced
functionality be required (e.g. technical rivalries with other colleges). Also
produced design and artwork.

\subsection*{UCSB Moodle Google Chrome Extension | {\footnotesize{Santa Barbara, CA}}}

Creator and Maintainer

\years{June 2012 - \ldots}

Produced opinionated UI modifications in the form of a Google Chrome extension
to University Courseware website. Additions include auto-login, minor usability
fixes, quick-menu, and psuedo-dashboard per course page. Audience of 60+ users.

\section*{Interests, Skills, and Hobbies}

\begin{packed_item}
\item \textbf{Languages} -- Scala, Java, Ruby, Object-oriented C/C++, and Python
\item \textbf{Operating Systems and Platforms}
        \begin{packed_item}
            \item Debian Server, Ubuntu Desktop and Server -- Debian Packaging and Administration
            \item Mac OS X -- Homebrew Recipes and \texttt{launchd}
            \item Android -- Native Java development.
            \item Heroku -- Ruby on Rails, Play Framework 2, and small scripts
        \end{packed_item}
    \item \textbf{Computer Networking} -- \texttt{iptables}, \texttt{pf}, and pfSense
    \item \textbf{IDE} -- IntelliJ IDEA, Eclipse, Unix
    \item \textbf{Text Editors} -- Vim
    \item \textbf{Databases} -- PostgreSQL and SQLite
    \item \textbf{Web Development} -- HTML5, SCSS and LESS, RoR, Play Framework 2, JavaScript, jQuery, Bootstrap, REST, and JSON
    \item \textbf{Misc. Skills} -- OpenStreetMap, \texttt{git}, and \LaTeX
\end{packed_item}

\section*{References}

\reference{Benjamin Price}{UCSB Housing and Residential Services}
{(805)893-5555}{bprice@housing.ucsb.edu}

%\hrule

%\hrule
% \section*{References}

% Available upon request.

%\vspace{1cm}
% \vfill{}
%\hrulefill
\end{multicols}

\begin{center}
{\scriptsize  Last updated: \today\- $\bullet$\- 
% ---- PLEASE LEAVE THIS BACKLINK FOR ATTRIBUTION AS PER CC-LICENSE
% Typeset in \href{http://nitens.org/taraborelli/cvtex}{\XeTeX}\\
% ---- FILL IN THE FULL URL TO YOUR CV HERE
Copies avaluable at \href{https://github.com/crazysim/cvtex}{https://github.com/crazysim/cvtex}}
\end{center}
\end{document}
