%-----------------------------------------------------------------------
% Anderson Konzen (anderson.konzen at gmail)
% Template of a CV typeset in XeTeX
%
% URL: https://github.com/anderkonzen/xetexcv
% DISCLAIMER: This template is provided for free and without any
%             guarantee that it will correctly compile on your
%             system if you have a non-standard configuration.
%
% This work is a fork from https://github.com/dartar/cvtex. This
% modified version is licensed under the same Creative Commons
% Attribution-ShareAlike license:
% http://creativecommons.org/licenses/by-sa/3.0/.
%
% (Please, keep a link to https://github.com/anderkonzen/xetexcv on
% your source file for attribution as per CC-license.)
%-----------------------------------------------------------------------

%!TEX TS-program = xelatex
%!TEX encoding = UTF-8 Unicode

\documentclass[10pt, letter]{article}

\def\myname{Nelson Chen}
\def\myemail{nelson@mindflakes.com}
% \def\myskype{crazysim1}
\def\mycellphone{(626) 723-3427}
\def\myaddress{Santa Barbara, CA}
\def\mygithub{nelsonjchen}


%% Loads the fontspec package, so we can easily select system fonts
%\usepackage{fontspec}
%% It implements some odds-and-ends features and improved functionality for broken or sub-standard LaTeX methods when using the XeTeX format. Already loads the fontspec package.
\usepackage{xltxtra}
%% Loads xcolor package, so we can use color names instead of rgb values
\usepackage{xcolor}

%-----------------------------------------------------------------------
% Document Layout
%-----------------------------------------------------------------------
%% Loads geometry package, so we can easily change page setup
\usepackage{geometry}
\geometry{letterpaper,         %% Specifies the paper size
  textwidth=7.5in,         %% Specifies the width of body (the text area)
  textheight=10.5in,        %% Specifies the height of body (the text area)
  marginparsep=7pt,        %% Modifies separation between body and marginal notes
  marginparwidth=.5in}     %% Modifies width of the marginal notes
\setlength\parindent{0in}  %% Amount of indentation at the first line of a paragraph.

\usepackage{multicol}

%-----------------------------------------------------------------------
% Fonts
%-----------------------------------------------------------------------
\setmainfont[Mapping={tex-text},Numbers={OldStyle},Ligatures={Common}]{Linux Libertine O}
\setsansfont[Mapping={tex-text},Color=AA0000]{Linux Biolinum O}
\setmonofont[Mapping={tex-text},Scale=0.8]{Menlo}

%-----------------------------------------------------------------------
% Custom Commands
%-----------------------------------------------------------------------
\chardef\&="E050  % Redefine '&' character
\newcommand{\html}[1]{\href{#1}{\scriptsize\textsc{[html]}}}
\newcommand{\pdf}[1]{\href{#1}{\scriptsize\textsc{[pdf]}}}
\newcommand{\doi}[1]{\href{#1}{\scriptsize\textsc{[doi]}}}
%% Configure margin years
\usepackage{marginnote}
\newcommand{\years}[1]{\marginnote{#1}}
\renewcommand*{\raggedleftmarginnote}{}  %% Define margin note alignment (in this case, justified text at the left margin)
%\setlength{\marginparsep}{7pt}  %% Already defined in geometry
\reversemarginpar  %% Margin notes in left side of the page

\renewcommand{\years}[1]{{\emph{#1}}}

% Reference
\newcommand{\reference}[4]{
\subsection*{#1 | {\footnotesize{#2}}}
#3 -- \href{mailto:#4}{\texttt{#4}}
}

\newenvironment{packed_enum}{
\begin{enumerate}
  \setlength{\itemsep}{1pt}
  \setlength{\parskip}{0pt}
  \setlength{\parsep}{0pt}
}{\end{enumerate}}

\newenvironment{packed_item}{
\begin{itemize}
  \setlength{\itemsep}{1pt}
  \setlength{\parskip}{0pt}
  \setlength{\parsep}{0pt}
}{\end{itemize}}

%-----------------------------------------------------------------------
% Headings
%-----------------------------------------------------------------------
%% Provides a set of commands for changing the font used for the various sectional headings
\usepackage{sectsty}
%% Provides various types of underlining that can stretch between words and be broken across lines
\usepackage[
  normalem  %% \em and \emph still produce normal italics
]{ulem}
%% Change font size of section headers
\sectionfont{\color[HTML]{AA0000}\mdseries\upshape\Large}
\subsectionfont{\mdseries\scshape\normalsize}
\subsubsectionfont{\mdseries\upshape\large}

%-----------------------------------------------------------------------
% PDF Setup
%-----------------------------------------------------------------------
\usepackage[
  xetex,            %% Use XeTeX backend
  colorlinks=true,  %% Colors the text of links and anchors
  breaklinks=true,  %% Allow links to break over lines
  pdftitle={{\myname} - Curriculum Vitae},
  pdfauthor={\myname}
]{hyperref}
\hypersetup{linkcolor=blue,filecolor=black,urlcolor=blue}

%-----------------------------------------------------------------------
%-----------------------------------------------------------------------
% DOCUMENT
%-----------------------------------------------------------------------
\begin{document}
{\LARGE \myname} - \emph{Software Engineer with an interest in rapid prototyping and productionization}\\
\hrule
\begin{multicols}{2}
% \vspace{0.0in}
{Phone: \mycellphone}\\[.01cm]
{Email: \href{mailto:\myemail}{\texttt{\myemail}}}\\[.05cm]
{GitHub: \href{http://www.github.com/\mygithub}{http://github.com/\mygithub}}
% {\large skype: \myskype}\\[.2cm]

% \myaddress
%\vfill
% \vspace{-0.2in}

%\hrule
\section*{Experience}

\subsection*{Mobivity Inc. (Acquired SmartReceipt Inc.) | {\footnotesize{Santa Barbara, CA}}}

Software Engineer

\years{December 2012 - \ldots}

Advised upon real-world development and systems administration practices.
Contributed to whole infrastructure refactoring with Chef and Salt Stack.
Contributed and maintained organization's main product offering's parsers and
internals. Developed new rapid \texttt{kitchen}, \texttt{vagrant}, and
automated testing workflow for Chef infrastructure development and application
development. Ported and applied same ``clean room sandbox'' workflow to
organization's legacy Rails application. Investigated feasibility and
experimented with full end-to-end Integration testing on Company's client-end
deployment. Produced customer map with static site generators, Google Map APIs
and custom styled maps. Supported and tutored fellow team members on AWS usage
security, and best practices. Migrated main customer-site-deployed software to
version control best practices.  Observed business, administration, and
financial practices.

\subsection*{Because of Hope (Nonprofit 501(c)3) | {\footnotesize{Santa Barbara, CA}}}

Web Engineer

\years{2012 - \ldots}

Maintained existing website and VPS. Drafted and implemented plans to help
organization reduce hosting costs, increase website performance, reduce
maintenance, and provide effortless scaling using Amazon Web Services.
Reimplemented website using static site generators and modern front-end
frameworks. Continued maintainence and update of content. Satisfying ``Help
poor children in Uganda'' directive in \texttt{vim}.

\subsection*{UCSB Associated Students | {\footnotesize{Santa Barbara, CA}}}

Technical Committee Vice Chair

\years{2009 - 2012}

Managed technical matters of committee activities including public website and
internal software and hardware. Designed, setup, configure, and maintained
a VMWare, Linux, Windows, and pfSense-based network, accompanying virtual
machines for file sharing and network traffic management, and devised and
improved upon workarounds for problematic network issues for LAN Parties for
150+ attendees for every quarter since Fall of 2009 until December 2012.

\subsection*{UCSB Housing and Residential | {\footnotesize{Santa Barbara, CA}}}

Network Consultant

\years{2009 - 2012}

Helped residents in UCSB Residential Housing get their devices online with
ResNet.  Developed tools to streamline configuration of these devices for
printing to shared printers and to automatically configure the network. Helped
physically install switches and hardware in dozens of locations on campus.
Maintained internal documentation with regards to networking and procedures
maintained. Assisted users one-on-one in troubleshooting any connectivity
issues both simple and complex. Produced automated graphical setup tools for
all major platforms to assist in configuration of 10,000 user systems with
Printing and Wi-Fi Configuration.

\section*{Selected Personal Projects and Contributions}

\subsection*{UC Santa Barbara Subreddit Design | {\footnotesize{Santa Barbara, CA}}}

Creator and Maintainer

\years{May 2012 - \ldots}

Produced most advanced Python, Ruby, and SASS framework and deployment tool for
developing preprocessed stylesheets and templated markdown for section of
website subscribed to by 3,000 users. Said framework is easy to deploy with
minimal configuration and utilizes accepted standard Python and
Ruby support tools. Framework is easy to extend should more advanced
functionality be required (e.g. technical rivalries with other student bodies
at other higher-learning institutions). Produced all design and artwork
from scratch and released under CC-BY-SA.


\section*{Education}

\years{2008 - 2012}

\textsc{Bachelor of Science}

Computer Science at the University of California, Santa Barbara

\section*{Interests, Skills, and Hobbies}

\begin{packed_item}
    \item \textbf{Languages} -- Python, Ruby, Scala, Go, and Java
    \item \textbf{Operating Systems and Platforms}
        \begin{packed_item}
            \item Ubuntu -- Debian Packaging and Administration
            \item Mac OS X -- Homebrew Recipes and \texttt{launchd}
            \item Android -- Native Java development
            \item Develoment and Operations -- Test-driven Chef, Salt Stack, Packer
            \item On-demand Compute and Storage -- Amazon Web Services and Rackspace Cloud
            \item Heroku -- Ruby on Rails, Play Framework 2, and automated scripts
        \end{packed_item}
    \item \textbf{Computer Networking} -- \texttt{iptables}, \texttt{pf}, and pfSense
    \item \textbf{IDE} -- IntelliJ IDEA and friends, Vim, Unix
    \item \textbf{Databases} -- PostgreSQL and SQLite
    \item \textbf{Web Development} -- HTML5, SCSS, LESS, nanoc, Ruby on Rails, Play Framework 2, JavaScript, jQuery, Zurb Foundation, Bootstrap, REST, and JSON
    \item \textbf{Project Management} -- Trello, HipChat
    \item \textbf{Version Control} -- \texttt{git}
    \item \textbf{Misc. Skills} -- AMPQ, Logstash, OpenStreetMap, AutoHotKey, Amazon Mechanical Turk, and \LaTeX
\end{packed_item}

% \section*{References}

% \reference{Benjamin Price}{UCSB Housing and Residential Services}
% {(805)893-5555}{bprice@housing.ucsb.edu}

% \reference{Natalie Lemonnier}{Because of Hope}
% {(818)620-8847}{natalie@becauseofhope.org}

% More references available upon request.

%\hrule

%\hrule
% \section*{References}

% Available upon request.

%\vspace{1cm}
% \vfill{}
%\hrulefill
\end{multicols}

\begin{center}
{\scriptsize  Last updated: \today\- $\bullet$\-
% ---- PLEASE LEAVE THIS BACKLINK FOR ATTRIBUTION AS PER CC-LICENSE
% Typeset in \href{http://nitens.org/taraborelli/cvtex}{\XeTeX}\\
% ---- FILL IN THE FULL URL TO YOUR CV HERE
Copies/Source/Attribution available at \href{https://github.com/nelsonjchen/cvtex}{https://github.com/nelsonjchen/cvtex}}
\end{center}
\end{document}
